\documentclass[10pt, a4paper,spanish]{article}

\usepackage[utf8]{inputenc}
\usepackage[spanish]{babel}

\usepackage[T1]{fontenc}

\usepackage[hmarginratio=1:1,top=32mm,columnsep=20pt]{geometry}
\usepackage[hang, small,labelfont=bf,up,textfont=it,up]{caption}

\usepackage{float}

\usepackage{amsmath}

\usepackage{graphicx}
\graphicspath{ {images/} }



\usepackage{abstract}
\renewcommand{\abstractnamefont}{\normalfont\bfseries}
\renewcommand{\abstracttextfont}{\normalfont\small\itshape}

\usepackage{minted}
\usepackage{float}
\RecustomVerbatimEnvironment{Verbatim}{BVerbatim}{}


\usepackage{titlesec}
\renewcommand\thesection{\Roman{section}}
\renewcommand\thesubsection{\Roman{subsection}}
\titleformat{\section}[block]{\large\scshape\centering}{\thesection.}{1em}{}
\titleformat{\subsection}[block]{\large}{\thesubsection.}{1em}{}

\usepackage{fancyhdr}
\pagestyle{fancy}
\fancyhead{}
\fancyfoot{}
\fancyhead[C]{ \today \ $\bullet$ Minería de Datos $\bullet$ Red de Hopfield}
\fancyfoot[RO]{\thepage}

%-------------------------------------------------------------------------------
%	TITLE SECTION
%-------------------------------------------------------------------------------

\title{\vspace{-15mm}\fontsize{24pt}{10pt}\selectfont\textbf{Red de Hopfield}}

\author{García Prado, Sergio}
\date{\today}

%-------------------------------------------------------------------------------

\begin{document}

	\maketitle

	\thispagestyle{fancy}


%-------------------------------------------------------------------------------
%	ABSTRACT
%-------------------------------------------------------------------------------

	\begin{abstract}
		\noindent Implementación de un ejemplo simple de una red de Hopfield diseñada para obtener el punto más cercano de entre dos dados previamente a una estructura en forma de forma de cubo.
	\end{abstract}

%-------------------------------------------------------------------------------
%	TEXT
%-------------------------------------------------------------------------------


  \section{Introducción}

    \paragraph{}
		Las redes de Hopfield son tipo de redes neuronales recurrente incialmente desarrolladas por John Hopfield. Estas redes son usadas como sistemas de Memoria asociativa con unidades binarias. Están diseñadas para converger a un mínimo local, a pesar de ello, la convergencia a uno de los patrones almacenados no está garantizada.

		\paragraph{}
		Las unidades de las redes Hopfield son binarias, es decir, solo tienen dos valores posibles para sus estados (Esto se consigue a partir del uso de una función signo). El valor se determina si las unidades superan o no un determinado umbral. Los valores posibles pueden ser 1 ó -1,


	\section{Implementación}

		\paragraph{}

		\begin{figure}[H]
			\centering
			\inputminted{octave}{../src/main.m}
		\end{figure}

		\begin{figure}[H]
			\centering
			\inputminted{octave}{../src/hopfield_learning.m}
		\end{figure}

		\begin{figure}[H]
			\centering
			\inputminted{octave}{../src/hopfield_working.m}
		\end{figure}

	\section{Resultados}

		\paragraph{}
		\begin{figure}[H]
			\centering
			\inputminted{bash}{../output.txt}
		\end{figure}

\end{document}
